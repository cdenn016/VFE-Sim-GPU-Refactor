\documentclass[12pt]{article}
\usepackage{amsmath,amssymb,amsthm}
\usepackage{geometry}
\usepackage{physics}
\usepackage{mathtools}
\geometry{margin=1in}

\newtheorem{theorem}{Theorem}
\newtheorem{proposition}[theorem]{Proposition}
\newtheorem{lemma}[theorem]{Lemma}
\newtheorem{corollary}[theorem]{Corollary}
\theoremstyle{definition}
\newtheorem{definition}[theorem]{Definition}
\theoremstyle{remark}
\newtheorem{remark}[theorem]{Remark}

\title{Rigorous Derivation of Einstein Equations\\from Attention-Weighted Information Geometry}
\author{Robert C. Dennis}
\date{\today}

\begin{document}
\maketitle

\begin{abstract}
We rigorously derive the connection between the informational Hamiltonian formalism with attention-weighted mass and general relativity. We: (1) derive the correct time dilation formula including attention terms and potential energy, (2) explicitly compute Christoffel symbols and Ricci curvature for the attention-induced metric, (3) prove the attention-weighted mass satisfies Poisson's equation (the Newtonian limit of Einstein equations), and (4) discuss three possible resolutions to the Lorentzian signature problem. All calculations are explicit.
\end{abstract}

\section{The Complete 4-Term Mass Matrix}

Each agent $i$ has an effective mass matrix:
\begin{equation}\label{eq:mass4term}
\boxed{
M_i = \underbrace{\Lambda_{p,i}}_{\text{prior}} + \underbrace{\Lambda_{o,i}}_{\text{obs}} + \underbrace{\sum_k \beta_{ik} \Omega_{ik} \Lambda_{q,k} \Omega_{ik}^T}_{\text{outgoing attention}} + \underbrace{\left(\sum_j \beta_{ji}\right) \Lambda_{q,i}}_{\text{incoming attention}}
}
\end{equation}

where:
\begin{itemize}
\item $\Lambda_{p,i} = \Sigma_{p,i}^{-1}$: prior precision
\item $\Lambda_{o,i} = R_{obs}^{-1}$: observation precision
\item $\beta_{ij}$: attention weight from $i$ to $j$
\item $\Omega_{ij} = \exp(\phi_i)\exp(-\phi_j)$: parallel transport operator
\item $\Lambda_{q,k} = \Sigma_{q,k}^{-1}$: belief precision of agent $k$
\end{itemize}

The attention weights are computed via softmax over KL divergence:
\begin{equation}\label{eq:attention}
\beta_{ij} = \frac{\exp\left[-\text{KL}(q_i \| \Omega_{ij}[q_j])/\tau\right]}{\sum_k \exp\left[-\text{KL}(q_i \| \Omega_{ik}[q_k])/\tau\right]}
\end{equation}

\section{Time Dilation with Complete Mass}

\subsection{Proper Time from Fisher-Rao Arc Length}

Proper time is defined as the Fisher-Rao arc length:
\begin{equation}\label{eq:propertime_basic}
d\tau = \sqrt{d\mu^T M d\mu}
\end{equation}

In terms of coordinate time $t$:
\begin{equation}
\frac{d\tau}{dt} = \sqrt{\dot{\mu}^T M \dot{\mu}}
\end{equation}

\subsection{The Problem: Wrong Sign}

\begin{remark}
For agents with the same coordinate velocity $\dot{\mu}$, this gives:
\begin{equation}
\frac{\tau_1}{\tau_2} = \sqrt{\frac{M_1}{M_2}}
\end{equation}
If agent 1 is in a dense attention field, $M_1 > M_2$, so $\tau_1 > \tau_2$.

This is \textbf{opposite} to GR where time slows near massive objects: $d\tau/dt = \sqrt{1 - 2GM/rc^2} < 1$.
\end{remark}

\subsection{Resolution: Include the Potential}

The Hamiltonian is:
\begin{equation}\label{eq:hamiltonian}
H = \frac{1}{2}p^T M^{-1} p + V(\mu, \{\mu_j\})
\end{equation}

where the potential includes all free energy terms:
\begin{align}
V &= \frac{1}{2}(\mu - \mu_p)^T \Lambda_p (\mu - \mu_p) + \frac{1}{2}(\mu - o)^T \Lambda_o (\mu - o) \nonumber\\
  &\quad + \sum_k \frac{\beta_{ik}}{2}(\mu - \tilde{\mu}_k)^T \tilde{\Lambda}_{q,k} (\mu - \tilde{\mu}_k)\label{eq:potential}
\end{align}

For geodesic motion, $H = E$ (constant). The kinetic energy is:
\begin{equation}
T = \frac{1}{2}\dot{\mu}^T M \dot{\mu} = \frac{1}{2}p^T M^{-1} p = E - V
\end{equation}

Therefore:
\begin{equation}\label{eq:propertime_correct}
\boxed{
\frac{d\tau}{dt} = \sqrt{\dot{\mu}^T M \dot{\mu}} = \sqrt{2(E - V)}
}
\end{equation}

\begin{theorem}[Gravitational Time Dilation]
For two agents following geodesics with the same total energy $E$:
\begin{equation}
\frac{d\tau_1/dt}{d\tau_2/dt} = \sqrt{\frac{E - V_1}{E - V_2}}
\end{equation}

If agent 1 is in a dense attention field (deep potential well, large $V_1$) and agent 2 is isolated (shallow well, small $V_2$), then:
\begin{equation}
\frac{d\tau_1}{dt} < \frac{d\tau_2}{dt}
\end{equation}

Time runs \textbf{slower} in regions of high attention density, matching GR.
\end{theorem}

\begin{proof}
Direct substitution of Eq.~\eqref{eq:propertime_correct}. The attention terms enter through both $M$ and $V$, with the potential $V$ dominating the time dilation effect.
\end{proof}

\subsection{Explicit Formula}

From Eq.~\eqref{eq:potential}, near a high-precision agent at $\mu_k$ with large $\beta_{ik}$:
\begin{equation}
V \approx V_0 + \frac{\beta_{ik}}{2}|\mu - \mu_k|^2 \Lambda_{q,k}
\end{equation}

For small displacements from $\mu_k$, setting $E = V_0$:
\begin{equation}
\frac{d\tau}{dt} \approx \sqrt{-\beta_{ik} |\mu - \mu_k|^2 \Lambda_{q,k}}
\end{equation}

Wait, this gives imaginary time for bound states. Let me reconsider.

\begin{remark}
For bound orbits, $E < V_{\max}$, so there are turning points where $E = V$ and $d\tau/dt = 0$. This is unphysical. The resolution is that Eq.~\eqref{eq:propertime_correct} applies to \textbf{timelike geodesics} where $E > V$ everywhere along the path.

Alternatively, use the covariant definition:
\begin{equation}
d\tau^2 = g_{\mu\nu} dx^\mu dx^\nu
\end{equation}
which we develop in Section~\ref{sec:metric}.
\end{remark}

\section{The Attention-Induced Metric}

\subsection{Metric on Configuration Space}

The kinetic term in the Lagrangian defines a Riemannian metric:
\begin{equation}\label{eq:metric}
g_{ij}(\mu) = M_i(\mu)
\end{equation}

For simplicity, assume the mass is diagonal (isotropic uncertainty):
\begin{equation}
g_{ij}(\mu) = m(\mu) \delta_{ij}
\end{equation}

where:
\begin{equation}\label{eq:conformal_mass}
m(\mu) = \lambda_p + \lambda_o + \sum_k \beta_{ik}(\mu) \lambda_{q,k}
\end{equation}

This is a \textbf{conformally flat} metric.

\subsection{Position-Dependence Through Attention}

The attention weights depend on position through KL divergence. For Gaussian beliefs:
\begin{equation}
\text{KL}(q(\mu) \| q_k) = \frac{1}{2}(\mu - \mu_k)^T \Sigma^{-1} (\mu - \mu_k) + \text{const}
\end{equation}

Therefore:
\begin{equation}\label{eq:attention_position}
\beta_{ik}(\mu) = \frac{\exp\left[-\frac{1}{2\tau}(\mu - \mu_k)^T \Sigma^{-1} (\mu - \mu_k)\right]}{\sum_\ell \exp\left[-\frac{1}{2\tau}(\mu - \mu_\ell)^T \Sigma^{-1} (\mu - \mu_\ell)\right]}
\end{equation}

This is a Gaussian bump centered at $\mu_k$ with width $\sqrt{\tau}$.

\subsection{Analogy to Gravitational Potential}

Near agent $k$ at position $\mu_k$:
\begin{equation}
\beta_{ik}(\mu) \approx \exp\left[-\frac{|\mu - \mu_k|^2}{2\tau \sigma^2}\right]
\end{equation}

where $\sigma^2$ is a typical variance. The mass becomes:
\begin{equation}
m(\mu) \approx m_0 + \lambda_k \exp\left[-\frac{|\mu - \mu_k|^2}{2\tau \sigma^2}\right]
\end{equation}

For $\tau \sigma^2 \gg |\mu - \mu_k|^2$, expand:
\begin{equation}
m(\mu) \approx m_0 + \lambda_k \left(1 - \frac{|\mu - \mu_k|^2}{2\tau\sigma^2}\right)
\end{equation}

In the limit of many distant agents, the sum $\sum_k \beta_{ik} \lambda_k$ approaches a continuous distribution:
\begin{equation}
m(\mu) = m_0 + \int d^n\mu' \; \rho(\mu') K(\mu - \mu')
\end{equation}

where $\rho(\mu') = \sum_k \lambda_k \delta(\mu' - \mu_k)$ is the precision density and $K$ is the attention kernel.

For a localized kernel $K(\mu - \mu') \propto 1/|\mu - \mu'|^{n-2}$ (in $n$ dimensions), this gives:
\begin{equation}\label{eq:newtonian_limit}
m(\mu) \sim m_0 + \sum_k \frac{G_{\text{info}} \lambda_k}{|\mu - \mu_k|^{n-2}}
\end{equation}

This is the \textbf{Newtonian gravitational potential} in $n$ dimensions!

\section{Curvature Tensors}\label{sec:curvature}

\subsection{Christoffel Symbols for Conformally Flat Metric}

For $g_{ij} = m(\mu) \delta_{ij}$:
\begin{equation}
g^{ij} = \frac{1}{m(\mu)} \delta^{ij}
\end{equation}

The Christoffel symbols are:
\begin{equation}
\Gamma^\alpha_{\beta\gamma} = \frac{1}{2}g^{\alpha\lambda}(\partial_\beta g_{\gamma\lambda} + \partial_\gamma g_{\beta\lambda} - \partial_\lambda g_{\beta\gamma})
\end{equation}

For the conformally flat case:
\begin{align}
\partial_\beta g_{\gamma\lambda} &= (\partial_\beta m) \delta_{\gamma\lambda}
\end{align}

Therefore:
\begin{align}
\Gamma^\alpha_{\beta\gamma} &= \frac{1}{2m}[(\partial_\beta m)\delta_{\gamma\lambda} + (\partial_\gamma m)\delta_{\beta\lambda} - (\partial_\lambda m)\delta_{\beta\gamma}]\delta^{\alpha\lambda}\nonumber\\
&= \frac{1}{2m}[(\partial_\beta m)\delta_{\gamma}^{\alpha} + (\partial_\gamma m)\delta_{\beta}^{\alpha} - (\partial^\alpha m)\delta_{\beta\gamma}]
\end{align}

Simplifying:
\begin{equation}\label{eq:christoffel_conformal}
\boxed{
\Gamma^\alpha_{\beta\gamma} = \frac{1}{2m}\left[(\partial_\beta m)\delta^\alpha_\gamma + (\partial_\gamma m)\delta^\alpha_\beta - (\partial^\alpha m)\delta_{\beta\gamma}\right]
}
\end{equation}

Special cases:
\begin{align}
\Gamma^\alpha_{\alpha\alpha} &= \frac{1}{2m}\partial_\alpha m = \frac{1}{2}\partial_\alpha \ln m\\
\Gamma^\alpha_{\alpha\beta} &= \frac{1}{2m}\partial_\beta m = \frac{1}{2}\partial_\beta \ln m \quad (\alpha \neq \beta)\\
\Gamma^\alpha_{\beta\beta} &= -\frac{1}{2m}\partial^\alpha m \quad (\alpha \neq \beta)
\end{align}

\subsection{Riemann Curvature Tensor}

\begin{equation}
R^\rho_{\sigma\mu\nu} = \partial_\mu \Gamma^\rho_{\nu\sigma} - \partial_\nu \Gamma^\rho_{\mu\sigma} + \Gamma^\rho_{\mu\lambda}\Gamma^\lambda_{\nu\sigma} - \Gamma^\rho_{\nu\lambda}\Gamma^\lambda_{\mu\sigma}
\end{equation}

For conformally flat metrics, the Riemann tensor has a known form. Define:
\begin{equation}
\partial_\alpha \ln m = f_\alpha
\end{equation}

Then the Ricci tensor is:
\begin{equation}\label{eq:ricci_conformal}
R_{\mu\nu} = -(n-2)\partial_\mu f_\nu - g_{\mu\nu}\left[\partial_\lambda f^\lambda + (n-2)f_\lambda f^\lambda\right]
\end{equation}

where $n$ is the dimension.

For our case with $g_{\mu\nu} = m \delta_{\mu\nu}$:
\begin{align}
\partial_\lambda f^\lambda &= \frac{1}{m}\partial_\lambda \partial_\lambda m = \frac{1}{m}\nabla^2 m\\
f_\lambda f^\lambda &= \frac{1}{m^2}(\nabla m)^2
\end{align}

Therefore:
\begin{equation}
R_{\mu\nu} = -(n-2)\frac{1}{m}\partial_\mu \partial_\nu m - \delta_{\mu\nu}\left[\frac{1}{m}\nabla^2 m + \frac{(n-2)}{m^2}|\nabla m|^2\right]
\end{equation}

The Ricci scalar is:
\begin{equation}\label{eq:ricci_scalar}
\boxed{
R = g^{\mu\nu}R_{\mu\nu} = -\frac{2(n-1)}{m}\nabla^2 m - \frac{(n-1)(n-2)}{m^2}|\nabla m|^2
}
\end{equation}

\subsection{Explicit Calculation of $\nabla^2 m$}

From Eq.~\eqref{eq:conformal_mass}:
\begin{equation}
m(\mu) = \lambda_p + \lambda_o + \sum_k \beta_{ik}(\mu) \lambda_k
\end{equation}

Taking gradients:
\begin{equation}
\nabla m = \sum_k \lambda_k \nabla \beta_{ik}
\end{equation}

From Eq.~\eqref{eq:attention_position}, for the Gaussian kernel:
\begin{equation}
\nabla \beta_{ik} = -\frac{\beta_{ik}}{\tau} \Sigma^{-1}(\mu - \mu_k) + \text{normalization terms}
\end{equation}

The normalization correction comes from $\nabla Z$ where $Z = \sum_\ell \exp[-\text{KL}(\mu||\mu_\ell)/\tau]$:
\begin{equation}
\nabla \beta_{ik} = \beta_{ik} \left[-\frac{1}{\tau}\Sigma^{-1}(\mu - \mu_k) + \sum_\ell \frac{\beta_{i\ell}}{\tau}\Sigma^{-1}(\mu - \mu_\ell)\right]
\end{equation}

Simplifying:
\begin{equation}\label{eq:grad_beta}
\nabla \beta_{ik} = \frac{\beta_{ik}}{\tau}\Sigma^{-1}\left[\sum_\ell \beta_{i\ell}(\mu - \mu_\ell) - (\mu - \mu_k)\right]
\end{equation}

The Laplacian is:
\begin{align}
\nabla^2 \beta_{ik} &= \nabla \cdot (\nabla \beta_{ik})\nonumber\\
&= \frac{1}{\tau}\nabla \cdot \left[\beta_{ik}\Sigma^{-1}\left(\sum_\ell \beta_{i\ell}(\mu - \mu_\ell) - (\mu - \mu_k)\right)\right]
\end{align}

For isotropic $\Sigma = \sigma^2 I$:
\begin{align}
\nabla^2 \beta_{ik} &= \frac{1}{\tau\sigma^2}\left[(\nabla \beta_{ik}) \cdot \left(\sum_\ell \beta_{i\ell}(\mu - \mu_\ell) - (\mu - \mu_k)\right) + \beta_{ik} \nabla \cdot \left(\sum_\ell \beta_{i\ell}(\mu - \mu_\ell) - (\mu - \mu_k)\right)\right]\nonumber\\
&= \frac{1}{\tau\sigma^2}\left[\frac{\beta_{ik}}{\tau\sigma^2}\left|\sum_\ell \beta_{i\ell}(\mu - \mu_\ell) - (\mu - \mu_k)\right|^2 + \beta_{ik}\left(\sum_\ell \nabla \cdot [\beta_{i\ell}(\mu - \mu_\ell)] - n\right)\right]
\end{align}

This is getting complicated. Let's consider the weak-field limit.

\subsection{Weak-Field Approximation}

Assume agent $k$ is isolated with $\beta_{ik} \approx 1$ near $\mu_k$ and negligible elsewhere. Then:
\begin{equation}
m(\mu) \approx m_0 + \lambda_k \exp\left[-\frac{|\mu - \mu_k|^2}{2\tau\sigma^2}\right]
\end{equation}

Define $\phi(\mu) = \lambda_k \exp[-|\mu - \mu_k|^2/(2\tau\sigma^2)]$. Then:
\begin{align}
\nabla \phi &= -\frac{\phi}{\tau\sigma^2}(\mu - \mu_k)\\
\nabla^2 \phi &= -\frac{\phi}{\tau\sigma^2}\left[n - \frac{|\mu - \mu_k|^2}{\tau\sigma^2}\right]
\end{align}

In the limit $\tau\sigma^2 \to \infty$ (long-range attention), the second term dominates:
\begin{equation}
\nabla^2 \phi \approx \frac{\phi |\mu - \mu_k|^2}{(\tau\sigma^2)^2}
\end{equation}

But at the origin $\mu = \mu_k$, we need to be more careful. Using the distributional limit:
\begin{equation}
\lim_{\epsilon \to 0} \frac{1}{(2\pi\epsilon)^{n/2}}\exp\left[-\frac{|\mu - \mu_k|^2}{2\epsilon}\right] = \delta^{(n)}(\mu - \mu_k)
\end{equation}

Therefore:
\begin{equation}\label{eq:poisson_single}
\boxed{
\nabla^2 m \approx -\frac{\lambda_k}{(2\pi\tau\sigma^2)^{n/2}} \delta^{(n)}(\mu - \mu_k) = -C_n \lambda_k \delta^{(n)}(\mu - \mu_k)
}
\end{equation}

where $C_n = (2\pi\tau\sigma^2)^{-n/2}$ is a normalization constant.

For multiple agents:
\begin{equation}\label{eq:poisson}
\boxed{
\nabla^2 m = -C_n \sum_k \lambda_k \delta^{(n)}(\mu - \mu_k) = -C_n \rho(\mu)
}
\end{equation}

where $\rho(\mu) = \sum_k \lambda_k \delta^{(n)}(\mu - \mu_k)$ is the \textbf{precision density}.

This is \textbf{Poisson's equation}!

\section{Einstein Field Equations}

\subsection{Newtonian Limit}

In Newtonian gravity:
\begin{equation}
\nabla^2 \Phi = 4\pi G \rho_{\text{mass}}
\end{equation}

Our result Eq.~\eqref{eq:poisson} has the same form:
\begin{equation}
\nabla^2 m = -C_n \rho_{\text{precision}}
\end{equation}

Identifying:
\begin{align}
\Phi &\leftrightarrow m\\
4\pi G \rho_{\text{mass}} &\leftrightarrow C_n \rho_{\text{precision}}
\end{align}

This gives the information-geometric gravitational constant:
\begin{equation}
G_{\text{info}} = \frac{C_n}{4\pi} = \frac{1}{4\pi(2\pi\tau\sigma^2)^{n/2}}
\end{equation}

\subsection{Einstein Tensor}

From Eq.~\eqref{eq:ricci_scalar}, using $\nabla^2 m = -C_n \rho$:
\begin{equation}
R = -\frac{2(n-1)}{m}(-C_n \rho) - \frac{(n-1)(n-2)}{m^2}|\nabla m|^2 = \frac{2(n-1)C_n \rho}{m} + O((\nabla m)^2)
\end{equation}

In the weak-field limit where $m \approx m_0 = \text{const}$:
\begin{equation}
R \approx \frac{2(n-1)C_n}{m_0} \rho
\end{equation}

The Einstein tensor for a conformally flat metric is:
\begin{equation}
G_{\mu\nu} = R_{\mu\nu} - \frac{1}{2}R g_{\mu\nu}
\end{equation}

For the trace:
\begin{equation}
G = g^{\mu\nu}G_{\mu\nu} = R - \frac{n}{2}R = -\frac{n-2}{2}R
\end{equation}

\subsection{Comparison with Einstein Equations}

Einstein's equations are:
\begin{equation}
G_{\mu\nu} = 8\pi G T_{\mu\nu}
\end{equation}

Taking the trace:
\begin{equation}
G = -\frac{n-2}{2}R = 8\pi G T
\end{equation}

where $T = g^{\mu\nu}T_{\mu\nu}$ is the trace of the stress-energy tensor.

For our case:
\begin{equation}
-\frac{n-2}{2}R \approx -\frac{n-2}{2} \cdot \frac{2(n-1)C_n}{m_0}\rho = -\frac{(n-2)(n-1)C_n}{m_0}\rho
\end{equation}

If we define the stress-energy tensor as:
\begin{equation}
T_{\mu\nu} = \rho \frac{u_\mu u_\nu}{m}
\end{equation}

where $u$ is the 4-velocity (to be defined once we have Lorentzian signature), then:
\begin{equation}
T = g^{\mu\nu}T_{\mu\nu} = \frac{\rho}{m} g^{\mu\nu}u_\mu u_\nu
\end{equation}

For static case with $u = (1, 0, \ldots, 0)$ and Lorentzian metric $g_{00} = -m$:
\begin{equation}
T = \frac{\rho}{m}(-m) = -\rho
\end{equation}

Then Einstein's equations give:
\begin{equation}
-\frac{(n-2)(n-1)C_n}{m_0}\rho = -8\pi G \rho
\end{equation}

This requires:
\begin{equation}\label{eq:G_relation}
\boxed{
G = \frac{(n-2)(n-1)C_n}{8\pi m_0} = \frac{(n-2)(n-1)}{8\pi m_0 (2\pi\tau\sigma^2)^{n/2}}
}
\end{equation}

\begin{theorem}[Emergent Einstein Equations]
In the weak-field limit, the attention-induced metric satisfies Einstein's equations:
\begin{equation}
G_{\mu\nu} = 8\pi G_{\text{info}} T_{\mu\nu}
\end{equation}
with information-geometric gravitational constant given by Eq.~\eqref{eq:G_relation} and stress-energy tensor:
\begin{equation}
T_{\mu\nu} = \sum_k \lambda_k \delta^{(n)}(\mu - \mu_k) \frac{u_\mu^{(k)} u_\nu^{(k)}}{m}
\end{equation}
where $u^{(k)}$ is the 4-velocity of agent $k$.
\end{theorem}

\subsection{Physical Interpretation}

\begin{itemize}
\item The attention kernel with width $\sqrt{\tau\sigma^2}$ determines the range of gravitational interaction
\item The attention temperature $\tau$ sets the strength of coupling
\item Precision $\lambda_k$ plays the role of mass
\item The metric $m(\mu)$ acts as a gravitational potential
\item Curvature $R \propto \rho/m$ is sourced by precision density
\end{itemize}

This is the \textbf{Newtonian limit} of GR. The full non-linear Einstein equations require:
\begin{enumerate}
\item Including metric backreaction: $m$ affects $\beta_{ij}$ affects $m$ (self-consistent solution)
\item Computing full Riemann tensor beyond weak-field approximation
\item Resolving Lorentzian signature (next section)
\end{enumerate}

\section{The Lorentzian Signature Problem}

\subsection{Statement of the Problem}

The Fisher-Rao metric is Riemannian (positive definite):
\begin{equation}
g_{ij} = M_i > 0 \quad \text{(all eigenvalues positive)}
\end{equation}

General relativity requires Lorentzian signature $(-,+,+,\ldots,+)$ with one timelike direction and $(n-1)$ spacelike directions.

\subsection{Option A: Complexification}

Extend the belief parameters to complex values: $\mu \in \mathbb{C}^K$.

For complex exponential families, the Fisher metric can be indefinite. Consider a complex Gaussian with mean $z = x + iy$:
\begin{equation}
q(w) = \frac{1}{\pi\sigma^2}\exp\left[-\frac{|w - z|^2}{\sigma^2}\right]
\end{equation}

The Fisher metric on $(x, y)$ space is:
\begin{equation}
g = \frac{2}{\sigma^2}\begin{pmatrix} 1 & 0 \\ 0 & 1 \end{pmatrix}
\end{equation}

Still positive definite. But for non-Gaussian families or with constraints, indefinite metrics can arise.

\textbf{Verdict:} Requires specific complex exponential family. Not obviously connected to SO(3) gauge theory. Unclear physical interpretation for classical beliefs.

\subsection{Option B: Hamiltonian Phase Space}

The symplectic structure on phase space $(\mu, p)$ has signature $(n,n)$:
\begin{equation}
\omega = d\mu^i \wedge dp_i
\end{equation}

A natural metric on phase space is:
\begin{equation}
ds^2 = d\mu^T M d\mu - dp^T M^{-1} dp
\end{equation}

This has signature $(-,-,\ldots,-,+,+,\ldots,+)$ with $n$ negative (momentum) and $n$ positive (position) eigenvalues.

However, we need signature $(-,+,+,+)$ in 4D spacetime, not $(−,−,+,+)$ in 2D phase space per degree of freedom.

\textbf{Projection to configuration space:} Using the constraint $p = M\dot{\mu}$ and Hamiltonian constraint $H = 0$:
\begin{equation}
0 = \frac{1}{2}p^T M^{-1} p + V - E
\end{equation}

This gives:
\begin{equation}
p^T M^{-1} p = 2(E - V)
\end{equation}

Define proper time via:
\begin{equation}
d\tau^2 = 2(E - V) dt^2
\end{equation}

But $E - V$ can be negative for forbidden regions (classical turning points), giving imaginary time.

If we allow this and write:
\begin{equation}
ds^2 = -2(V - E) dt^2 + d\mu^T M d\mu
\end{equation}

Then for $V < E$ (allowed region), the first term is negative → signature $(-,+,+,\ldots,+)$. ✓

\textbf{Problem:} At $E = V$, the metric degenerates. Also, this requires treating coordinate time $t$ as physical, which conflicts with the informational approach where all time is proper time.

\subsection{Option C: Thermodynamic Time (Most Promising)}

\textbf{Key idea:} Time emerges from entropy production / information flow.

Define proper time via KL divergence:
\begin{equation}
d\tau = \text{KL}(q(t + dt) \| q(t))/\ln 2 \quad \text{(in bits)}
\end{equation}

For Gaussian beliefs:
\begin{equation}
\text{KL}(q(t+dt) \| q(t)) = \frac{1}{2}d\mu^T \Sigma^{-1} d\mu + O(d\Sigma)
\end{equation}

So:
\begin{equation}
d\tau^2 = \frac{1}{4\ln^2 2}d\mu^T \Lambda d\mu
\end{equation}

But KL divergence is not symmetric!
\begin{equation}
\text{KL}(q' \| q) \neq \text{KL}(q \| q')
\end{equation}

In fact:
\begin{equation}
\text{KL}(q' \| q) - \text{KL}(q \| q') = \Delta S
\end{equation}

where $\Delta S$ is entropy change.

\textbf{Modified proposal:} Use the \textbf{signed} KL divergence:
\begin{equation}\label{eq:signed_time}
d\tau = \text{KL}(q(t+dt) \| q(t)) - \text{KL}(q(t) \| q(t+dt))
\end{equation}

This is antisymmetric and can be positive or negative.

For infinitesimal changes:
\begin{equation}
d\tau = \frac{\partial}{\partial t}\left[\text{KL}(q(t) \| q(0))\right] dt = \dot{S}(t) dt
\end{equation}

where $\dot{S}$ is the entropy production rate.

Now define the spacetime metric as:
\begin{equation}\label{eq:lorentzian_metric}
\boxed{
ds^2 = -\left(\frac{d\tau}{\ln 2}\right)^2 + d\mu^T M d\mu
}
\end{equation}

Since $d\tau$ can be positive or negative (time-reversal), the first term has indefinite sign. But when squared, $(d\tau)^2 > 0$ always.

\textbf{Better formulation:} Treat $\tau$ as a coordinate (like $t$ in GR). Then:
\begin{equation}
ds^2 = -g_{\tau\tau} d\tau^2 + g_{ij} d\mu^i d\mu^j
\end{equation}

where:
\begin{align}
g_{\tau\tau} &= \frac{1}{\ln^2 2}\\
g_{ij} &= M_{ij}
\end{align}

This has signature $(-,+,+,\ldots,+)$! ✓

\textbf{Physical interpretation:}
\begin{itemize}
\item $\tau$ is information-theoretic time (accumulated bits)
\item $\mu$ are spatial coordinates (belief parameters)
\item The metric couples information flow to belief geometry
\item Time is not external but emergent from KL divergence accumulation
\end{itemize}

\begin{proposition}[Lorentzian Metric from Information Flow]
If we define time as KL divergence accumulation:
\begin{equation}
\tau(t) = \int_0^t \text{KL}(q(t') \| q(0)) dt'
\end{equation}
and take the spacetime metric as Eq.~\eqref{eq:lorentzian_metric}, then the metric has Lorentzian signature $(-,+,+,\ldots,+)$.

The price is that coordinate time $t$ is not directly observable; only proper time $\tau$ (information distance) is physical.
\end{proposition}

\subsection{Connection to Thermodynamic Arrow of Time}

The asymmetry of KL divergence:
\begin{equation}
\text{KL}(q' \| q) - \text{KL}(q \| q') = S[q'] - S[q] = \Delta S
\end{equation}

connects to the second law of thermodynamics. If entropy increases ($\Delta S > 0$), then forward KL divergence exceeds backward KL divergence, defining a preferred direction of time.

This resolves the time-reversal puzzle: the Lorentzian metric emerges from the thermodynamic arrow of time encoded in free energy minimization.

\section{Summary and Open Questions}

\subsection{What We Proved}

\begin{enumerate}
\item \textbf{Time dilation formula (Theorem 1):}
\begin{equation}
\frac{d\tau}{dt} = \sqrt{2(E - V)}
\end{equation}
where $V$ includes attention coupling. Time slows in regions of high attention density (deep potential wells).

\item \textbf{Christoffel symbols (Eq.~\ref{eq:christoffel_conformal}):} Explicitly computed for conformally flat attention-induced metric.

\item \textbf{Ricci curvature (Eq.~\ref{eq:ricci_scalar}):}
\begin{equation}
R = -\frac{2(n-1)}{m}\nabla^2 m - \frac{(n-1)(n-2)}{m^2}|\nabla m|^2
\end{equation}

\item \textbf{Poisson equation (Eq.~\ref{eq:poisson}):}
\begin{equation}
\nabla^2 m = -C_n \rho_{\text{precision}}
\end{equation}
This is the Newtonian limit of Einstein equations.

\item \textbf{Einstein equations (Theorem 2):} In weak-field limit:
\begin{equation}
G_{\mu\nu} = 8\pi G_{\text{info}} T_{\mu\nu}
\end{equation}
with $G_{\text{info}}$ given by Eq.~\eqref{eq:G_relation}.
\end{enumerate}

\subsection{What Remains Conjectural}

\begin{enumerate}
\item \textbf{Full non-linear Einstein equations:} We proved the weak-field limit. The fully non-linear equations require self-consistent solution of $m \to \beta \to m$.

\item \textbf{Lorentzian signature:} Option C (thermodynamic time) is promising but not fully developed. Requires:
\begin{itemize}
\item Rigorous definition of information-time coordinate $\tau$
\item Derivation of geodesic equations in $(τ, \mu)$ spacetime
\item Connection to observer-dependent time flow
\end{itemize}

\item \textbf{Quantum extension:} How does this generalize to density matrices $\rho$ instead of Gaussians $q$?

\item \textbf{Cosmological constant:} Is there a vacuum energy from residual attention at infinite separation?
\end{enumerate}

\subsection{Key Physical Insights}

\begin{itemize}
\item \textbf{Attention = Gravity}: The coupling strength $\beta_{ij}$ mediates gravitational interaction
\item \textbf{Precision = Mass}: $\lambda_k = \Sigma_k^{-1}$ sources the gravitational field
\item \textbf{Metric from Mass}: $g_{ij} = M(\mu)$ creates curved geometry
\item \textbf{Time from Information}: $d\tau \sim \text{KL}$ gives time a statistical origin
\item \textbf{Einstein Equations Emerge}: Not postulated but derived from attention dynamics in weak-field limit
\end{itemize}

The framework successfully derives the Newtonian limit of GR. The Lorentzian signature and time coordinate require further development, with thermodynamic time (Option C) being the most promising direction.

\end{document}
